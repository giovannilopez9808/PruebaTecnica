\documentclass[../main.tex]{subfiles}
\begin{document}
\section*{Sección 1}
Realizando un estudio sobre el comportamiento de las variables del documento dado, se encontraron las siguientes características.
\begin{itemize}
    \item No existe una corrrelación fuerte contemplando todas las variables del documento con el parametro Rango score. Ya que, el valor de las correlaciones son menores a 0.2.
    \item Existe una mayor probabilidad que los clientes clasificados como BB abandonen el prodcuto en comparación a los clientes clasificados como otros y AA.(Tabla 1)
    \item Puede que exista un sezgo en el modelo debido a que existe una diferencia clara cuando no existe dispersión en el rango disponible. Esto puede deberse a que sea un nuevo cliente. (Tabla 2)
    \item A mayor cantidad de productos contratados, la probabilidad de abandono se reduce.
    \item A mayor antigüedad, la probabilidad de abandono se reduce considerablemente.
\end{itemize}
\begin{table}[htb]
    \centering
    \begin{tabular}{llll}
        \hline
        Rango\_score   & Otros  & AA     & BB     \\ \hline
        0\% a 10\%     & 15.565 & 21.089 & 10.393 \\
        10.1\% a 20\%  & 24.272 & 28.649 & 19.464 \\
        20.1\% a 30\%  & 20.813 & 21.826 & 19.419 \\
        30.1\% a 40\%  & 16.681 & 14.108 & 16.522 \\
        40.1\% a 50\%  & 11.961 & 8.121  & 13.204 \\
        50.1\% a 60\%  & 6.73   & 3.94   & 9.591  \\
        60.1\% a 70\%  & 2.845  & 1.66   & 6.35   \\
        70.1\% a 80\%  & 1.014  & 0.533  & 3.545  \\
        80.1\% a 90\%  & 0.111  & 0.075  & 1.375  \\
        90.1\% a 100\% & 0.009  &        & 0.138  \\ \hline
    \end{tabular}
    \caption{Distribución de rango score para cada clasificación.}
\end{table}
\begin{table}[htb]
    \tiny
    \begin{tabular}{lllllllll}
        \hline
        Rango\_score   & Sin dispersión & 0.1 a 99.9 & 100 a 4999.9 & 5000 a 9999.9 & 10000 a 19999.9 & 20000 a 39999.9 & 40000 a 59999.9 & 60000 o más \\ \hline
        0\% a 10\%     & 28.467         & 1.299      & 9.053        & 7.228         & 10.672          & 15.87           & 17.97           & 16.554      \\
        10.1\% a 20\%  & 29.927         & 3.896      & 14.341       & 15.562        & 20.539          & 25.282          & 28.654          & 27.423      \\
        20.1\% a 30\%  & 24.088         & 6.494      & 14.608       & 17.411        & 20.66           & 22.215          & 22.821          & 23.001      \\
        30.1\% a 40\%  & 10.219         & 18.182     & 14.268       & 16.851        & 17.45           & 16.097          & 14.488          & 15.388      \\
        40.1\% a 50\%  & 2.19           & 14.286     & 13.495       & 15.086        & 13.503          & 10.502          & 8.445           & 9.359       \\
        50.1\% a 60\%  & 3.65           & 6.494      & 11.995       & 12.145        & 9.043           & 5.883           & 4.516           & 4.949       \\
        60.1\% a 70\%  & 0.73           & 12.987     & 10.22        & 8.707         & 5.144           & 2.833           & 2.189           & 2.258       \\
        70.1\% a 80\%  & 0.73           & 18.182     & 7.538        & 5.001         & 2.299           & 1.085           & 0.773           & 0.89        \\
        80.1\% a 90\%  &                & 15.584     & 3.95         & 1.848         & 0.653           & 0.223           & 0.145           & 0.175       \\
        90.1\% a 100\% &                & 2.597      & 0.532        & 0.161         & 0.038           & 0.01            &                 & 0.003       \\ \hline
    \end{tabular}
    \caption{Distribución de rango score para cada distribución del promedio disponible del cliente.}
\end{table}
\begin{table}[htb]
    \centering
    \begin{tabular}{lllll}
        \hline
        Rango\_score   & 0      & 1      & 2      & Más de 2 \\ \hline
        0\% a 10\%     & 19.048 & 10.696 & 12.487 & 11.369   \\
        10.1\% a 20\%  & 20     & 17.282 & 21.345 & 22.278   \\
        20.1\% a 30\%  & 6.667  & 16.616 & 19.82  & 22.131   \\
        30.1\% a 40\%  & 1.905  & 14.787 & 16.092 & 17.709   \\
        40.1\% a 50\%  & 4.762  & 13.009 & 12.507 & 12.52    \\
        50.1\% a 60\%  & 7.619  & 10.706 & 8.719  & 7.756    \\
        60.1\% a 70\%  & 18.095 & 8.253  & 5.446  & 4.148    \\
        70.1\% a 80\%  & 12.381 & 5.57   & 2.706  & 1.717    \\
        80.1\% a 90\%  & 9.524  & 2.705  & 0.849  & 0.365    \\
        90.1\% a 100\% &        & 0.376  & 0.029  & 0.007    \\ \hline
    \end{tabular}
    \caption{Distribución del rango score para la cantidad de productos contratados.}
\end{table}
\begin{table}[htb]
    \centering
    \begin{tabular}{lllllll}
        \hline
        Rango\_score   & 6 o menos & 7 a 12 & 13 a 24 & 25 a 60 & 61 a 120 & 121 o más \\ \hline
        0\% a 10\%     & 4.855     & 4.812  & 5.608   & 7.796   & 11.248   & 21.27     \\
        10.1\% a 20\%  & 7.321     & 7.856  & 9.583   & 15.541  & 24.45    & 31.679    \\
        20.1\% a 30\%  & 7.286     & 10.503 & 13.27   & 19.672  & 24.578   & 22.217    \\
        30.1\% a 40\%  & 8.907     & 13.348 & 17.208  & 19.963  & 17.986   & 12.794    \\
        40.1\% a 50\%  & 12.127    & 16.64  & 19.056  & 16.601  & 11.205   & 6.652     \\
        50.1\% a 60\%  & 15.822    & 17.993 & 16.632  & 11.151  & 5.862    & 2.981     \\
        60.1\% a 70\%  & 18.521    & 16.003 & 11.181  & 5.792   & 2.791    & 1.436     \\
        70.1\% a 80\%  & 16.147    & 9.405  & 5.319   & 2.52    & 1.356    & 0.733     \\
        80.1\% a 90\%  & 7.951     & 3.107  & 1.932   & 0.894   & 0.501    & 0.23      \\
        90.1\% a 100\% & 1.063     & 0.332  & 0.211   & 0.069   & 0.023    & 0.007     \\ \hline
    \end{tabular}
    \caption{Distribución del rango score para cada rango de antigüedad.}
\end{table}

Con los puntos antes mencionados podemos delimitar a retener a las personas que tengan una antigüedad entre 7 a 12 y tengan a lo más un producto contratado. Esto debido a que pasando al rango de 13 a 24 años o que tengan al menos dos productos contratado la probabilidad de abandono baja considerablemente.
\end{document}
